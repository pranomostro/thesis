% !TeX root = ../main.tex
% Add the above to each chapter to make compiling the PDF easier in some editors.

\chapter{Introduction}\label{chapter:introduction}

3 pages

Software often has has faults: ways in which the actual behavior of
the software diverges from the intended or specifie behavior. Computer
scientists have devised different strategies for finding and removing
bugs: Formal software verification, code reviews, and different types
of testing: unit testing, which tests the behavior of small software
modules (such as classes), integration testing, which shows how well the
software works in the environment it is used in, and regression testing,
which shows developers whether they have introduced new bugs after the
last change to the software.

%CANDO: add sources for the cost of development & also hopefully long
%execution times for test suites.

Regression testing takes up a significant portion of development cost,
and the resulting test suites can grow quite significantly in size and
execution time, which hinders development speed and increases costs.

To mitigate the costs and runtimes of regression test suites, different
strategies have been devised: test case selection, which selects a subset
of tests for the current execution of tests, test suite reduction, which
permanently deletes a subset of tests from the test suite, and test case
priorization, which changes the order of test execution to maximize the
amount of faults that is found early in the test suite execution.

\cite{cruciani2019scalable} presents a new family of test suite
reduction algorithms, called the FAST algorithms (first developed in
\cite{miranda2018fast}), and compares them to the algorithms presented
in \cite{chen2010adaptive}, as well as the greedy algorithm presented
in \cite{rothermel2001prioritizing}. They implement 4 algorithms
from the FAST family, 2 from the ART family and the greedy algorithm,
and compare the different algorithms on 10 different test programs
and their test suites. They compare the performance of the test suite
reduction algorithms on 3 different variables: test suite reduction,
fault detection loss, and runtime.

This work attempts to replicate their findings using the data from 6
additional projects, as well as adding random test case selection as
a baseline test suite reduction method to compare the other methods to
(following \textbf{Recommendation 8} from \cite{khan2018systematic}).

This work then attempts to determine how different test suite reduction
strategies compare to each other in terms of time performance, fault
detection loss and magnitude of reduction.

%\section{Section}
%Citation test~\parencite{latex}.
%
%\subsection{Subsection}
%
%See~\autoref{tab:sample}, \autoref{fig:sample-drawing}, \autoref{fig:sample-plot}, \autoref{fig:sample-listing}.
%
%\begin{table}[htpb]
%  \caption[Example table]{An example for a simple table.}\label{tab:sample}
%  \centering
%  \begin{tabular}{l l l l}
%    \toprule
%      A & B & C & D \\
%    \midrule
%      1 & 2 & 1 & 2 \\
%      2 & 3 & 2 & 3 \\
%    \bottomrule
%  \end{tabular}
%\end{table}
%
%\begin{figure}[htpb]
%  \centering
%  % This should probably go into a file in figures/
%  \begin{tikzpicture}[node distance=3cm]
%    \node (R0) {$R_1$};
%    \node (R1) [right of=R0] {$R_2$};
%    \node (R2) [below of=R1] {$R_4$};
%    \node (R3) [below of=R0] {$R_3$};
%    \node (R4) [right of=R1] {$R_5$};
%
%    \path[every node]
%      (R0) edge (R1)
%      (R0) edge (R3)
%      (R3) edge (R2)
%      (R2) edge (R1)
%      (R1) edge (R4);
%  \end{tikzpicture}
%  \caption[Example drawing]{An example for a simple drawing.}\label{fig:sample-drawing}
%\end{figure}
%
%\begin{figure}[htpb]
%  \centering
%
%  \pgfplotstableset{col sep=&, row sep=\\}
%  % This should probably go into a file in data/
%  \pgfplotstableread{
%    a & b    \\
%    1 & 1000 \\
%    2 & 1500 \\
%    3 & 1600 \\
%  }\exampleA
%  \pgfplotstableread{
%    a & b    \\
%    1 & 1200 \\
%    2 & 800 \\
%    3 & 1400 \\
%  }\exampleB
%  % This should probably go into a file in figures/
%  \begin{tikzpicture}
%    \begin{axis}[
%        ymin=0,
%        legend style={legend pos=south east},
%        grid,
%        thick,
%        ylabel=Y,
%        xlabel=X
%      ]
%      \addplot table[x=a, y=b]{\exampleA};
%      \addlegendentry{Example A};
%      \addplot table[x=a, y=b]{\exampleB};
%      \addlegendentry{Example B};
%    \end{axis}
%  \end{tikzpicture}
%  \caption[Example plot]{An example for a simple plot.}\label{fig:sample-plot}
%\end{figure}
%
%\begin{figure}[htpb]
%  \centering
%  \begin{tabular}{c}
%  \begin{lstlisting}[language=SQL]
%    SELECT * FROM tbl WHERE tbl.str = "str"
%  \end{lstlisting}
%  \end{tabular}
%  \caption[Example listing]{An example for a source code listing.}\label{fig:sample-listing}
%\end{figure}
