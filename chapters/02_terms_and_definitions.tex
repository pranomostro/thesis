% !TeX root = ../main.tex
% Add the above to each chapter to make compiling the PDF easier in some editors.

\chapter{Terms and Definitions}\label{chapter:terms}

This paper uses some terms and abbreviations that are common in the
literature on test suite reduction, but not used much elsewhere.

The \textbf{System Under Test (SUT)} is a software system being tested
using a test suite.

The \textbf{Test Suite Reduction (TSR)} is a metric that evaluates the
magnitude of the reduction of a test suite. Given a test suite $T=\{t_1,
\dots t_n\}$ and a reduced test suite $T' \subset T$, the TSR of the
test suite is determined using the formula

$$\hbox{TSR}(T, T')=100*\frac{|T|-|T'|}{|T|}$$

which gives the percentage of tests removed from the test suite.

Everything else equal, higher values for TSR are better.

The abbreviation TSR also sometimes stands for the technique of test
suite reduction, i.e. the permanent removal of redundant test cases
from a test suite. In this text, "TSR" will only stand for the metric,
and the technique will be written out as "test suite reduction".

The \textbf{Fault Detection Loss (FDL)} is a metric that evaluates the
amount of faults detected by a reduced test suite. Let $F$ be the faults
detected by $T$, and $F'$ be the faults detected by $T'$, then the fault
detection loss is calculated as such:

$$\hbox{FDL}(F, F')=100*\frac{|F|-|F'|}{|F|}$$

which returns the percentage of faults not detected anymore.

In general, lower FDL values are better. A related metric (e.g. used in
\cite{khan2016survey} or \cite{yoo2012regression}) is the Fault Detection
Effectiveness (FDE), defined as $\hbox{FDE}=1-FDL(F, F')$.

\textbf{Mutation testing} (first described by \cite{budd1980mutation})
is the practice of intentionally introducing faults (e.g. changing
comparisons, moving lines or deleting function bodies) into a SUT in
order to discover which tests are activated by the faults introduced.

Mutation testing is often often automated.
