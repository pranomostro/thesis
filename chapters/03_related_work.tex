% !TeX root = ../main.tex
% Add the above to each chapter to make compiling the PDF easier in some editors.

\chapter{Related Work}\label{chapter:related}

%CANDO: Write about regression testing, model-based testing where
%tests are selected from automatically generated test suites?

5 pages

The literature on handling large test suites is comprehensive, and
this summarization will only cover small parts of it. For an older,
but comprehensive overview dividing the field into test-suite
reduction, test case selection and test case priorization, see
\cite{yoo2012regression}. A newer overview focused exclusively on
test suite reduction is \cite{khan2016survey}, which attempts to
create a full taxonomy of TSR frameworks/tools and specifically their
implementions. \cite{khan2018systematic} focuses on the algorithmic
approach (i.e. greedy/clustering/searching/hybrid) of the TSR system
and gives recommendations for future the evaluation of TSR methods.

The methods for improving the runtimes of large test suites discussed
in the literature can be divided into three related approaches
(\cite{yoo2012regression}): Test suite minimization (here called test
suite reduction), test case priorization, and test case selection.

%Distinction: Temporary or permanent, adequate or inadequate
%ordered or unordered, recency-aware and recency-unaware.

%Move this to terms & definitions instead?

Let $T=\{t_1, t_2, \dots, t_n\}$ be a set of $n$ tests, and a
set of requirements $R=\{r_1, \dots, r_m\}$. These requirements
can take very different forms: they can correspond to coverage of
lines/branches/functions, parts of an explicit specification that must
be tested, runtimes of individual tests, or faults that were discovered
in the past.

Let $m: T \rightarrow \mathcal{P}(R)$ a function that maps test cases
to the requirements they test (one test can test multiple requirements,
but any requirement needs only one test case to be fulfilled, i.e. there
can be no requirements that need 2 or more tests to be fulfilled).

\section{Test Case Priorization}

Test case priorization attempts to find a permutation $T' \in S_{T}$
($S_{T}$ being the set of all permutations of T) of test cases that
cover as many requirements as early as possible.

More formally, the test case priorization problem is to find a permutation
$T' \in S_{T}$ so that

$$ \forall T'' \in S_{T}: T' \neq T'' \land \forall i \in 1..n: |\bigcup_{j=1}^{i} m(T'(j))| \geq |\bigcup_{j=1}^{i} m(T''(j))| $$

%CANDO Maybe use APFD here instead?

Test case priorization subsumes test suite reduction: Given an ordering
$T'$ of test cases, it is easy to select the first $o$ test cases that
together fulfill all requirements.

\section{Test Case Selection}

Test case selection attempts to temporarily find a subset of tests that
maximize the requirements for a set of of recent changes to the software.

\cite{yoo2012regression} formulate the problem of test case selection as

"\textit{Given}: The program, P, the modified version of P, P' and a test suite, T. \\
\textit{Problem}: Find a subset of T, T', with which to test P'."

Test case selection usually assumes a high degree of transparency and
the availability of a lot of information about the SUT, such as execution
graphs, coverage information, and sometimes information about the running
times of individual tests.

\section{Test Suite Reduction}

Test suite reduction, on the other hand, attempts to permanently reduce
the size of the test suite by removing redundant tests. The distinguishing
feature from test case selection is that it doesn't take into account
recent changes.

One can now distinguish two cases:

%CANDO: create subsubsections here?

Generally, in test case reduction two different cases are distinguished:
adequate and inadequate reduction.

In adequate test suite reduction, the goal is to find the smallest subset
of $T$ so that all $R$ are satisfied:

$$T_{r}=\hbox{argmin}_{T_{r}} |T_{r} \subset T: R=\bigcup_{t_{r} \in T_{r}} m(t_{r})|$$

This goal, however, is rarely fulfilled. Often, the objective is to
keep the resulting test suite as small as possible, since reaching
this optimum is computationally hard (as researchers have remarked
(\cite{khan2016survey}), it is equivalent to the NP-complete set cover
problem). %CANDO: cite michael & David 1979 here.

Instead, researchers attempt to reduce the size of test suites further,
while still keeping running times small.

Other possible goals include to reduce the total runtime of the reduced
test suite, but due to the absence of information about test runtimes,
this goal has been studied less. %CANDO: [citation needed]

Adequate reduction needs the requirements $R$ to be present at the time
of reduction to judge whether all requirements have been fulfilled.

The second case of test suite reduction is the inadequate one. An
inadequate test suite reduction fixes the size of the reduce test suite
to a budget $B=|T_{r}|$. This cannot guarantee that all requirements
are completely fulfilled, but often reduces runtimes significantly.
%CANDO: perhaps find a source for this statement.

Inadequate reduction can work without access to the requirements.
For example, the FAST family needs only the content of the testcases to
perform a reduction. However, inadequate reduction gives no guarantees
about the performance of the resulting reduced test suite.

\cite{khan2018systematic} distinguishes 4 different types of test
suite reduction approaches: Greedy selection, Clustering, Searching,
and Hybrid methods.

\subsection{Greedy Selection}

Greedy approaches, first introduced by \cite{rothermel2001prioritizing}
(in the context of test case priorization), work by selecting test cases
based on a coverage criterion $C=\{c_1, \dots, c_{\hbox{ncov}}\}$.

\cite{khan2018systematic} describe the fully general case of greedy-based
test suite reduction, in which a test case is first selected using
the coverage criterion and a heuristic, and then removed from the test
set and added to the set of the reduced test suite, until the coverage
criterion is completely fulfilled by the reduced test suite.

Two simple heuristics, described more in detail later, are the total
heuristic (choosing the test with the highest amount of coverage from
the test set) and the additional heuristic (choosing the test with the
highest amount of coverage yet not covered by the reduced test suite).

\subsection{Clustering}



\subsection{Searching}

\subsection{Hybrids}
