% !TeX root = ../main.tex
% Add the above to each chapter to make compiling the PDF easier in some editors.

\chapter{Future Work}\label{chapter:futurework}

The literature on test suite reduction is very comprehensive,
\cite{khan2018systematic} identify 4320 papers published in the field
of test suite reduction between 1993 and 2016. This makes it difficult
to assess which ideas have already been evaluated, and which haven't.
Still, we attempt to identify some possible further lines of work that
haven't been explored as much as desirable yet.

The first possible like of work we identify is further replication
attempts of existing papers, and comparisons with random baseline
algorithms.  Replication attempts are useful because they can unveil
methodological issues in existing papers (such as discovering bugs in
the source code used, statistical mistakes and distorted data), but are
usually neglected because of low academic payoff. %CANDO: citation needed

The second possible line of work is large-scale comparisons of many
different test suite reduction algorithms on comprehensive datasets.
A good example for this kind of work is \cite{hemmati2010achieving},
but it uses a relatively small dataset with 2 test suites (although
the test suites themselves are very large), one containing 4 and
the other 15 faults. They examine 300 different approaches to test
suite reduction. It would be useful (though very expensive, both
computationally and time-wise) to collect a large number of test suite
reduction approaches presented in the literature, and compare them to
each other on several metrics.

A possible line of work that is not directly a novel approach to
test suite reduction is the collection of fault and coverage data for
open-source project to create bigger datasets on which to test novel
approaches to test suite reduction. \cite{yoo2012regression}
state that "subject programs from SIR account for almost 60\% of
the subjects of empirical studies observed in the regression testing
techniques literature" (SIR being the software infrastructure repository
\cite{dosupporting2005}). The programs and their test suites in SIR are
relatively small, and include only one language. It might be useful to
provide a combined and extended database of projects with test suites,
coverage and fault information and faults (\cite{cruciani2019scalable}
uses SIR, as well as programs from Defects4j \cite{just2014defects}).

Test suite reduction and similar methods are often motivated by stating
that test suites in the industry are often very large and have long
runtimes. It would be useful to have a collection of cases which show
runtimes of very large test suites in the real world, to provide sources
for these types of motivations.
