% !TeX root = ../main.tex
% Add the above to each chapter to make compiling the PDF easier in some editors.

\chapter{Summary}\label{chapter:summary}

In this paper, we motivated and formulated the problem of handling large
test suites: As test suites for regression testing and model-based testing
become more comprehensive, their running times are often prohibitive to
effective software development. We examined three different approaches
to this problem formulated in the literature: test case priorization,
test case selection, and test suite reduction. For test suite reduction,
we in turn presented and shortly discussed a categorization of different
algorithms: greedy selection algorithms, clustering-based algorithms,
search-based algorithms, and hybrids.

We then explained 7 algorithms examined in the paper
\cite{cruciani2019scalable}: four algorithms from the FAST family (FAST++,
based on the k-means++ algorithm; FAST-CS, based on coreset discovery;
FAST-pw, based on locality-sensitive signature hashing; and FAST-all,
which is based on FAST-pw, but uses random sampling for selecting test
cases instead of Jaccard distance); two algorithms from the ART family
(ART-D, which maximizes Jaccard distance between prioritized test cases;
and ART-F, which maximizes the Manhattan distance); and the greedy
additional algorithm (which selects the test case with the highest amount
of yet uncovered program components). We also described random selection
of test cases, which was not described in the original paper.

We then formulated four research questions, which outlined our attempt
to replicate the findings in \cite{cruciani2019scalable}, examining
the metrics FDL, TSR, total runtime performance, and performance of
the different algorithms against a random baseline. We described our
process of gathering independent coverage and fault data and executing
the algorithms on this data.

We then collected and presented both descriptive and inductive statistical
measurements on the data collected. We showed the mean, median, and
standard deviation of FDL, TSR, and total runtime performance for all
algorithms, and ranked the algorithms on FDL, TSR and total runtime
performance using the Kruskal-Wallis test. We found that our results largely
replicated the original results from \cite{cruciani2019scalable}, although
we observed differences especially in total runtime performance.

We described both threats to validity for our results, and presented
ideas for future work in the area.
